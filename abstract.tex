
%The light-matter interactions of plasmonic nanostructures are of great importance to [whom?].
%For plasmonic Janus particles in particular, which are established as an often-used tool in active matter research, the [???] is two-fold. [...]
%Metal-coated Janus particles have established themselves as a valuable tool in active matter research \cite{MONA-flow-fields} and related fields, owing to their plasmonically active nanostructures which allow for both excellent visibility in dark-field applications, as well as for efficient heating and, consequently, motility. \cite{MONA-thermotaxis, MONA-photon-nudging-1} 
%With light-matter interactions enhanced through their plasmonically active nanostructures, metal-coated Janus particles allow for both excellent visibility in dark-field applications as well as efficient heating and, consequently, motility \cite{MONA-thermotaxis, MONA-photon-nudging-1}, making them a valuable tool for active matter research. \cite{MONA-flow-fields}
%In recent years, theoretical predictions have been made of counter-intuitive optically enabled behaviours of such particles. \cite{Ilic2017}
%Nonetheless, these light-matter interactions are not yet thoroughly understood; in particular, there does not yet exist a proven method of inferring the out-of-image-plane orientation of a spherical Janus particle from its dark-field image. 
%Nonetheless, these light-matter interactions are not yet thoroughly understood.

%We present an experimental method for the orientation-dependent scattering spectroscopy of such anisotropic microscale particles. 
% $\SI{1}{\micro\meter}$ PS + $\SI{50}{\nano\meter}$ Au
%We analysed the scattering behaviour of \mbox{1µm PS + 50 nm Au} Janus Particles, resolving for wavelength, direction of illumination and scattering angle.
%under dark-field illumination and reproduced the results by finite-elements simulations. 
%We find well-discernible spectral features, particularly in the NIR range, that appear or disappear, depending on the orientation of the JP. 
%Further, we reproduced the experimental results by finite-element simulations.
%These findings may be worked into a method of determining the notoriously elusive out-of-plane orientation of spherical JPs, possibly even in real-time.

%We identify spectral markers, particularly in the near infrared range, that depend heavily on the orientation of the Janus particle; a result that may enable a new method of determining the notoriously elusive [to do: reference] out-of-plane orientation of spherical JPs, possibly even in real-time.
Plasmonic Janus particles consist of dielectric core particles with a thin metallic cap on one side and are widely used in active matter research. \cite{MONA-flow-fields} 
The plasmonic cap enhances optical scattering and absorption, allowing for self-propulsion through temperature gradients as well as efficient trapping and tracking. \cite{MONA-thermotaxis, MONA-photon-nudging-1} 
The asymmetry of such a particle gives rise to surface plasmon modes whose excitation is sensitive to the angle at which the particle is illuminated. 
Even though the angle of illumination strongly influences the particle's scattering response, the optical properties of such metallic caps have hardly been investigated. 

We probe the light scattering of individual micrometre-sized, spherical, Au-coated Janus particles by means of Selective Illumination Multiplexed Fourier Plane Spectroscopy. 
This novel method allows us to explore microparticles' scattering characteristics resolved for wavelength, angle of illumination and scattering angle. 

In addition, we supplement our experimental results with finite-element simulations and correlate spectral markers to orientation-dependent surface plasmon modes. 
This additional information on the correlation of angular and spectral information could pave the way for new methods of orientation detection. They also shed new light on the interaction of such spherically capped particles with light inducing forces and torques. \cite{Ilic2017, BA}
